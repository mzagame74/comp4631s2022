\documentclass[11pt]{article}
\usepackage{cite}

\begin{document}

\title{Mobile App Programming II Bibliography}
\author{Matt Zagame}
\maketitle

The internet of things (IoT) and its implementation into wearable smart devices
has proven to provide a host of benefits to human lifestyle and health. IoT
devices are ushering in a wave of innovation by allowing us to connect nearly
any device that we might use on a daily basis to a pervasive smart network that
can be controlled from our fingertips. Also, new wearable and embeddable tech
makes staying connected even easier and provides great health care benefits.
For example, the average user is able to monitor their own health, and health
care providers have access to more tools (such as AR) to offer better care.
So as long as IoT and wearables remain useful to most users, this emerging
technology could certainly be here for the long-term. ~\cite{iot_1}.

Web3 is being coined as the evolution of the internet. However, its aim is to
take a step back from the over-centralization of the current web 2.0 to the
early days of the internet which was more decentralized. Essentially, striking
a balance between web 1.0 and the current internet that is primarily controlled
by large corporations. This is done by layering blockchain technology onto the
web to store and transfer data on an encrypted network that is not owned
nor operated by any one entity. Currently it is too early to tell whether this
will be possible to implement in a truly decentralized way and on such a large
scale. However, if implemented, it could forever change the way information
moves across the internet. ~\cite{web3_1}

\bibliography{references}
\bibliographystyle{plain}
\end{document}
